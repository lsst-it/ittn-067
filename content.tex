\section{Introduction}

The following documents lists all the conditions and requirements that a platform or system needs to met before it is deployed in Rubin's network. 

Upon its installation in Rubin's network, it is assumed that the DevOps team will take control of it, will handle its administration, and will provide some basic troubleshooting. 

\section{Requirements}

In order to integrate any platform, system of computer with Rubin's network, it must met the following requirements. 

\subsection{Hardware}

As a general rule, no hardware will be installed in Rubin. Exceptions will be handled by the DevOps team and several conditions must be met, such as:

\begin{itemize}
    \item Equipment must come in rack form. 
    \item It must have dual power supply
    \item Unless special situations it should not exceed 2RUs
    \item It is recommended it has dual network connection
    \item It must have IPMI/BMC/IDRAC connection and license must be already purchased if needed.
    \item 
\end{itemize}

Any special networking requirement must be discussed with the DevOps team. Some network protocols or features are not available.

\subsection{Software}

Software must run in Rubin’s  Kubernetes cluster and be compliant with policies required for pods to run. Devops team will provide access to a development cluster to test deployments. If due to technical restrictions it is not possible to run the software in Kubernetes cluster, the DevOps team will decide where to run it. 

Under very particular conditions it will be allowed to install servers in Rubin's network. However, if a server is allowed, it must be wiped out and reinstalled from scratch, hence instructions to build the software must be included in the deliverables. 

All that will be run in Rubin's network must adhere to DM standard published at https://developer.lsst.io/

Software must be delivered as a repo in Github. If the vendor does not have a Github accounts, the DevOps team can provide a repo. 

If the the software requires the authentication or validation of a user to make use of it, the system must integrate with the authentication backend of Rubin based on IPA (ldap). The DevOps team will create a special user with to bind to IPA. 

The system must integrate to Rubin’s Monitoring platform. This could be accomplished by publishing an API, Snmp or any other method that would allow Rubin’s platform to pull a state. If the system is meant to trigger alarms, notifications, etc, it must integrate with Rubin’s notification system. 

\subsection{Deliverable}

The deliverable must include:

\begin{itemize}
    \item Documentation of design, components, usage and APIs when applicable.
    \item  Detailed description of monitoring and notifications configurations. 
    \item If a server is meant to be installed in Rubin's network, instruction to build the software from scratch must be included. 
    \item Troubleshooting procedures
    \item Backup strategy
    \item Patching strategy	
\end{itemize}
	

